\section{Motivating Example} \label{sec:motivating_example}


 %COMMENT: \ls{The case for SPL needs to be elaborated:  we must argue that the various GNOME subprojects share sufficient code to define a product base.  If too many of them are too different, SPL would be inappropriate.}
%\todo{This sentence is problematic:  we shoudl discuss. Done}
%\eb{from Leandro: I've rewritten this sentence by providing details of how GNOME could benefit from using SPLE principles as mass customisation and a common platform.}

\textbf{ORIGINAL VERSION}

%to delete
\sout{The open-source GNOME project\footnote{https://wiki.gnome.org/Projects} encompasses a large portfolio of individual programs evolve as independently as possible from the rest. These programs share features, but because they are separately developed, their constituent features cannot be easily reused across its portfolio to provide mass customization, at least without much manual effort. The combination of mass customisation and a common platform, principles of Software Product Line Engineering (SPLE)~\cite{Pohl2005}, would allows to GNOME team reuses a common base of technology and, at the same time, to bring out products tailored individual customers. Without a common platform and a software development process base on mass customisation, it may be more difficult to the GNOME project provides customized products and effectively manage the commonality and variability of its features.}

%\textcolor{blue}{In the dynamic landscape of open-source development, projects such as GNOME \footnote{https://wiki.gnome.org/Projects} harbor an extensive collection of independently evolving programs. While these programs share common features, their separate development trajectories hinder seamless reuse across the entire portfolio for mass customization, necessitating manual efforts. Principles of Software Product Line Engineering (SPLE)\cite{Pohl2005}, embodying mass customization and a common platform, could empower the GNOME team to reuse a foundational technology base while tailoring products to individual customers. However, the adoption of SPLE by GNOME faces a significant obstacle. The intricate web of programs within GNOME, developed independently over time, lacks a common platform and a software development process grounded in mass customization. Consequently, providing customized products and effectively managing the commonality and variability of features becomes a great challenge.}

\textcolor{blue}{In the dynamic landscape of open-source development, projects such as GNOME \footnote{https://wiki.gnome.org/Projects} harbor an extensive collection of independently evolving programs. While these programs share common features, their separate development trajectories hinder seamless reuse across the entire portfolio for mass customization, necessitating manual efforts. Principles of SPLE\cite{Pohl2005}, embodying mass customization and a common platform, could empower the GNOME team to reuse a foundational technology base while tailoring products to individual customers. However, the adoption of SPLE by GNOME faces a significant obstacle. The intricate web of programs within GNOME, developed independently over time, lacks a common platform and a software development process grounded in mass customization. Consequently, providing customized products and effectively managing the commonality and variability of features becomes a great challenge.}
%to delete
\sout{The GNOME  project is a natural candidate for SPL, but the significant reengineering investment of time and resources have prevented it from adopting SPL. \FOUNDRY is transformative because it can be used to reduce this cost.}

\textcolor{blue}{However, the adoption of SPLE by GNOME faces a significant obstacle. The intricate portifolio of programs within GNOME, developed independently over time, lacks a common platform and a software development process grounded in mass customization. Consequently, providing customized products and effectively managing the commonality and variability of features becomes a great challenge.}

%to delete
\sout{We show how the GNOME team could use \prodscalpel to quickly generate a product line. Suppose project collaborators want to build a product line in the domain of text editors. This product line would allow GNOME to produce text editors that augment its current text editor, \emph{GEdit}\footnote{https://wiki.gnome.org/Apps/Gedit}, with additional features. Since they have decided to augment Gedit, GNOME team would select it as the product base, the shared substrate of a product line that, for \FOUNDRY, serves the host for transplanted features. Assume that the GNOME team targets the following three features (1) $\texttt{side-panel}$, (2) $\texttt{split pane}$, and (3) $\texttt{presentation}$. They then identify two donors from which to transplant these features:\emph{NEdit}\footnote{https://sourceforge.net/projects/nedit/}, a multi-purpose text editor that is not part of the GNOME portfolio, and \emph{Evince}\footnote{https://wiki.gnome.org/Apps/Evince}, a document viewer for multiple document formats that is part of GNOME, not not an editor. }

\textcolor{blue}{To elucidate, consider a scenario where the GNOME team aims to swiftly generate a product line within the domain of text editors. This endeavor involves became their current text editor, \emph{GEdit}\footnote{https://wiki.gnome.org/Apps/Gedit}, in a product line with the possiblity to transplant additional features. GEdit is selected as the product base, a shared substrate that, in the context of Foundry, serves as the host for transplanted features.}

\sout{Suppose the team targets three features—(1) $\texttt{side-panel}$, (2) $\texttt{split pane}$, and (3) $\texttt{presentation}$. To enrich GEdit product line with these features, two donors are identified: \emph{NEdit}\footnote{https://sourceforge.net/projects/nedit/}, a multi-purpose text editor external to GNOME, and \emph{Evince}\footnote{https://wiki.gnome.org/Apps/Evince}, a document viewer within GNOME but not an editor.}
 
\textcolor{blue}{Here's where \FOUNDRY and \prodscalpel} can be used to automate the re-engineering process. \FOUNDRY iteratively and incrementally re-engineers the codebase for SPL. In the donor, engineers need to demarcate all feature entry points to transplant; a single annotation is sufficient for \prodscalpel to extract a feature. 

To prepare the host, GNOME engineers use \prodscalpel to extract a product base from an existing system by removing all features not shared across all products within the built product line. Then, the engineers must annotate the host to indicate the implantation point for each target feature, or “organ” using transplantation nomenclature. GNOME engineers then run \prodscalpel on these inputs, once per feature, with 

\begin{quote}
    $\texttt{./prodScalpel \textemdash \textemdash seeds\_file}$: \textit{ The file which contains the seeds for Genetic Programming(GP) algorithm.} 

    $\texttt{\textemdash \textemdash donor\_folder}$: \textit{The path to the donor source code.} 
    
    $\texttt{\textemdash \textemdash host\_target}$: \textit{The file in host that contains the insertion point of the transplant.}
    
    $\texttt{\textemdash \textemdash donor\_target}$: \textit{The file in the donor that contains the core function.}
    
    $\texttt{\textemdash \textemdash workspace}$: \textit{The path to the workspace of the transplant.}
    
    $\texttt{\textemdash \textemdash core\_function\_target}$: \textit{The file which contains all feature entry points.}
    
     $\texttt{\textemdash \textemdash  host\_project}$: \textit{The path to the product base source code.}
     
   \end{quote} 

This command automatically extracts all of the specified feature's source code and its dependencies, or “over-organ” using transplantation nomenclature. More operational parameters are available at the project: webpage~\cite{ProjectWebpage}.

 \begin{figure*}[t]
	\centering \includegraphics[width=\textwidth]{images/incremental_ST4.png}
	\caption{Product derivation process using the \FOUNDRY approach. \textit{\prodscalpel transplants three features, in sequence, into the GEdit's product base to derive a new text editor after three iterations of organ transplantation.} }
	\label{fig:incremental_pd}
\end{figure*} 

Figure 1 illustrates all transplantation iterations performed to generate a new product. It shows a new text editor derived from the transplant of features from different donors and using GEdit as a product base, using feature models to represent each donor system and the product base evolution. \textcolor{blue}{Importantly, the donor systems in this context are not product lines; we are using feature models to map some of their features, making it easier to understand what has been extracted from the donors' codebase.} \prodscalpel first transplants the \emph{side-panel} feature, extracted from GEdit itself. This transplantation showcases \prodscalpel's ability to transplant features into a product base from the same codebase. Next, \prodscalpel transplants the \emph{split\_pane} feature from NEdit, highlighting its ability to integrate features from diverse sources seamlessly into a unified product base. Finally, \prodscalpel transplants the \emph{presentation} feature from GNOME's Evince renderer, further highlighting its ability to integrate features from diverse sources seamlessly into a unified product base.

\textcolor{blue}{It is imperative to emphasize ProdScalpel's agnostic nature towards the order of transplantations. ProdScalpel is designed to operate independently of the transplantation sequence, providing practitioners with the flexibility to tailor the process according to their specific requirements.  However, it is noteworthy that while ProdScalpel accommodates any transplantation sequence, the choice of order may impact the resulting substrate's compactness by influencing interdependencies between transplanted features. }

\FOUNDRY facilitates transplanting features from any program into a product line, opening the door to large scale feature reuse. Open-source projects, like GNOME, are an especially promising source of code for \FOUNDRY, so long as the donors and target hosts share compatible licenses.

