\section{Conclusions and Future Work} \label{sec:conclusion} 

In this work, we propose an approach, \FOUNDRY, and a tool, \prodscalpel, that use software transplantation to speed conversion to and maintenance of SPL.
Both approach and the tool have been validated through case studies. 
We generated two products through the transplantation of features extracted from three real-world systems into two different product bases. 
Moreover, we performed an experiment with SPL experts to compare our approach with manual effort.
We showed significant time effort improvements when using \prodscalpel. 
The tool accomplished product line generation process by migrating two features 4.8 times faster than the mean time spent by participants who were able to finish the experiment within the timeout.

We argue that the migration to SPL transplantation-based in contrast to a configuration-based software product line makes it easier to use in practice.
Our approach improves SPL maintainability by physically separating features from the product base.  
\FOUNDRY can be used both for \emph{extractive} or \emph{reactive} product line migration, as well as as a systematic Clone\&own strategy to specialize existing products.
\FOUNDRY avoids code duplication of feature implementations, while preserving feature behavior. 
It can automatically propagate feature changes. 
That is, it provides solutions for problems often cited in reengineering of systems into SPL literature.

Our evaluation studies provide initial evidence to support the claim SPLE 
using software transplantation, is a feasible and, indeed, promising direction for SPL research and practice.
However, more studies are needed to provide more evidence for generalisability of our approach, and to investigate its applicability in an industrial context.

